\documentclass[a4paper]{article}
\title{CHE260---Thermodynamics \& Heat Transfer}
\author{Jonah Chen}

\usepackage[utf8]{inputenc}

\usepackage{braket}
\usepackage{physoly}
\usepackage{currfile}
\usepackage{gensymb}
\usepackage{amssymb}
\usepackage{pgf,tikz,pgfplots}
\usepackage{mathrsfs}
\usepackage{textcomp}
\usetikzlibrary{arrows}
\numberwithin{equation}{section}
\pgfplotsset{compat=1.16}

\begin{document}
    \maketitle
    \tableofcontents
	\section{Basics}
	Fourier's law relates heat flux to temperature
	\begin{equation}
		\dot q=-k\nabla T
	\end{equation}

	Newton's Law of Cooling is the fundamental law of convection.
	\begin{equation}
		\dot q=h(T-T_\infty)
	\end{equation}

	Thermal radiation is
	\begin{equation}
		\dot q=\sigma A(T^4-T_\infty^4)
	\end{equation}

    \section{Conduction}
	\subsection{Heat Equation}
	The heat equation is a statement of the first law (conservation of energy), which describes conduction.

	\begin{equation}
		(\partial_t-\alpha\nabla^2)T=0
	\end{equation}
	
	\subsection{Steady State}
	In steady state, $\partial_t T=0$, so the heat equation reduces to Laplace's equation for temperature $\nabla^2 T=0$.
	Energy Balance for something:	
	\begin{equation}
		\rho c_p\partial_t T=-\frac{1}{A}\partial_x(\dot qA)
	\end{equation}

	1-dimensional Heat Equation	
	\begin{align}
		(\partial_t-\alpha\partial_x^2)T=0
	\end{align}
	where $\alpha:=\frac{k}{\rho c_p}$, with units of m$^2/$s.
	\begin{itemize}
		\item High $k$ --- Material conducts heat well.
		\item High $\rho c_p$ --- Material stores energy well.
	\end{itemize}

	In cylinderical coordinates (in the r direction), 
	\begin{equation}
		(\partial_t-\frac{\alpha}{r}\partial_r(r\partial_r)) T = 0
	\end{equation}
	
	In spherical coordinates (in the r direction),
	\begin{equation}
		(\partial_t-\frac{\alpha}{r^2}\partial_r(r^2\partial_r)) T = 0
	\end{equation}
	
	Common Solutions:
	Steady state solution for cartesian coordinates is
	\begin{equation}
		T = mx+b
	\end{equation}

	\subsection{Thermal Resistance}

	When dealing with steady state behavior for multiple connected mediums, it is useful to define a quantity called thermal resistance with the property that 
	\begin{equation}
		\dot Q=\frac{T_1-T_2}{R}
	\end{equation}

	The calculated values for resistances are
	\begin{itemize}
		\item For conduction, $R=\frac{L}{kA}$.
		\item For convection, $R=\frac{1}{hA}$.
		\item For radiation, $R=\frac{1}{\varepsilon\sigma A(T^2-T_\infty^2)(T-T_\infty)}$.
	\end{itemize}

	For two mediums connected in series,
	\begin{equation}
		R_T=R_1+R_2
	\end{equation}

	For two mediums connected in parallel,
	\begin{equation}
		R_T=\frac{R_1R_2}{R_1+R_2}
	\end{equation}

	It is also useful to define an overall heat transfer coefficient $U=\frac{1}{R_TA}$
	\begin{equation}
		\dot Q=UA(T_1-T_2)
	\end{equation}

	The rvalue is $L/k$, hence
	\begin{equation}
		\dot Q=\frac{\Delta T}{R_{value}}\times A
	\end{equation}
	which is in english units. $L$ is in feet and $k$ is in BTU per h square feet Fahernhit.

	The critical radius of insulation is when the resistance is stationary. When increasing the radius, we are increacing conductive resistance but decreasing convective resistance. 
	\begin{equation}
		r_{crit}=\frac{k}{h}
	\end{equation}

	For a very long fin with perimeter $P$, conduction and convection coefficients $k$ and $h$ in an enviroment at $T_\infty$, (we are also assuming steady state somehow, but not sure how)
	\begin{equation}
		(k\partial_x(A\partial_x)-hP)T=-hPT_\infty
	\end{equation}
	Assume $A,k,P$ are constant. Define $\Theta=T-T_\infty$ and $a^2=\frac{hP}{kA}$. Then,
	\begin{equation}
		(\partial^2_x-a^2)\Theta=0
	\end{equation}
	Using boundary conditions of $\Theta(0)=\Theta_b$ and $\Theta(\infty)=0$ (from the definition of $T_\infty$), the unique solution is
	\begin{equation}
		\Theta=\Theta_be^{-ax}
	\end{equation}
	The rate of heat loss is just equal to the heat entering the base of the fin. Using fourier's law,
	\begin{equation}
		\dot Q=\sqrt{hPkA}(T_b-T_\infty)
	\end{equation}

	For a finitely long fin, assume the heat transfer to the air at the end is negligiable (adiabatic). Hence, $\partial_x\Theta|_{x=L}=0$.	
	\begin{equation}
		\Theta=\Theta_b\frac{\cosh(a(L-x))}{\cosh(aL)}
	\end{equation}
	By similar means, the \textbf{rate of heat loss} for this case is
	\begin{equation}
		\dot Q=\dot Q_\infty2
	\end{equation}
	Where $\dot Q_\infty$ is the rate of heat loss assuming the fin is infinitely long.

	If there is some heat lost to the air, we can correct the length $L_C=L+\frac{A}{P}$

	To define the \textbf{fin efficiency}, we first define the ``ideal'' fin to be a fin with infinite heat conductivity, and 
	\begin{equation}
		\eta_{fin}:=\frac{\dot Q_{fin}}{\dot Q_{fin,max}}=\frac{\tanh(aL)}{aL}
	\end{equation}
	For $L>>a$, $\tanh(aL)$ is approximately $1$.

	We define the \textbf{fin effectiveness} as
	\begin{equation}
		\varepsilon_{fin}:=\frac{\dot Q_{fin}}{\dot Q_{no\,fin}}=\sqrt{\frac{kP}{hA}}
	\end{equation}
	To maximize effectiveness,
	\begin{itemize}
		\item Maximize $P/A$, use a lot of small fins. 
		\item Maximize $k$, use copper or something similar.
		\item Don't use fin for liquids because $h$ is big.
		\item Generally, we use fins if $\epsilon\gtrapprox2$.
	\end{itemize}
	
	\subsection{Lump Bodies}
	The lump capacitance is defined as $L=\nabla^2T$. For a lump at uniform temperature, $L=0$ and the heat equation reduces to 
	\begin{equation}
		\partial_tT=0
	\end{equation}
	This assumption is valud when the object is small or have high thermal conductivity.

	The stored thermal energy of the lump is being lost to the surrounding through convection. So $\dot E_{store}+\dot Q_{conv}=0$. Defining a time constant $\tau=\frac{\rho vc_p}{hA}$ and a dimensionless temperature $\Theta=\frac{T-T_\infty}{T_0-T_\infty}$,
	\begin{equation}
		\Theta=e^{-t/\tau}
	\end{equation}

	The dimensionless constant $hL/k$ is called Biot's number
	\begin{equation}
		\frac{T_1-T_2}{T_2-T_\infty}=B
	\end{equation}
	where $k$ is thermal conductivity of the \textbf{solid}. If $\mathrm{Bi}>>1$, then the temperature drop within the body is much greater than temperature drop outside the body. If $\mathrm{Bi}<<1$, then the temperature drop within the body is much smaller than temperature drop outside the body. Hence, the body can be considered as a lump.

	
\end{document}
