\documentclass[a4paper]{article}
\title{CHE260---Thermodynamics \& Heat Transfer}
\author{Jonah Chen}

\usepackage[utf8]{inputenc}

\usepackage{braket}
\usepackage{physoly}
\usepackage{currfile}
\usepackage{gensymb}
\usepackage{amssymb}
\usepackage{pgf,tikz,pgfplots}
\usepackage{mathrsfs}
\usepackage{textcomp}
\usetikzlibrary{arrows}
\numberwithin{equation}{section}
\pgfplotsset{compat=1.16}

\begin{document}
    \maketitle
    \tableofcontents
	
    \section{Conduction}

	\subsection{Fourier's Law (relates heat flux to temperature)}
	\begin{equation}
		\dot q=-k\frac{\partial T}{\partial x}
	\end{equation}


	\subsection{Heat Equation}
	The heat equation is a statement of the first law (conservation of energy), which describes conduction.

	\begin{equation}
		(\partial_t-\alpha\nabla^2)T=0
	\end{equation}
	
	In steady state, $\partial_t T=0$, so the heat equation reduces to Laplace's equation for temperature $\nabla^2 T=0$.
	Energy Balance for something:	
	\begin{equation}
		\rho c_p\partial_t T=-\frac{1}{A}\partial_x(\dot qA)
	\end{equation}

	1-dimensional Heat Equation	
	\begin{align}
		(\partial_t-\alpha\partial_x^2)T=0
	\end{align}
	where $\alpha:=\frac{k}{\rho c_p}$, with units of m$^2/$s.
	\begin{itemize}
		\item High $k$ --- Material conducts heat well.
		\item High $\rho c_p$ --- Material stores energy well.
	\end{itemize}

	In cylinderical coordinates (in the r direction), 
	\begin{equation}
		(\partial_t-\frac{\alpha}{r}\partial_r(r\partial_r)) T = 0
	\end{equation}
	
	In spherical coordinates (in the r direction),
	\begin{equation}
		(\partial_t-\frac{\alpha}{r^2}\partial_r(r^2\partial_r)) T = 0
	\end{equation}
	
	Common Solutions:
	Steady state solution for cartesian coordinates is
	\begin{equation}
		T = mx+b
	\end{equation}

	\subsection{Thermal Resistance}

	When dealing with steady state behavior for multiple connected mediums, it is useful to define a quantity called thermal resistance with the property that 
	\begin{equation}
		\dot Q=\frac{T_1-T_2}{R}
	\end{equation}

	The calculated values for resistances are
	\begin{itemize}
		\item For conduction, $R=\frac{L}{kA}$.
		\item For convection, $R=\frac{1}{hA}$.
		\item For radiation, $R=\frac{1}{\varepsilon\sigma A(T^2-T_\infty^2)(T-T_\infty)}$.
	\end{itemize}

	For two mediums connected in series,
	\begin{equation}
		R_T=R_1+R_2
	\end{equation}

	It is also useful to define an overall heat transfer coefficient $U=\frac{1}{R_TA}$
	\begin{equation}
		\dot Q=UA(T_1-T_2)
	\end{equation}

   
	\section{Convection}
	Newton's Law of Cooling:
	\begin{equation}
		\dot q=h(T-T_\infty)
	\end{equation}

	
	
	
\end{document}
