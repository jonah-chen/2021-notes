\documentclass[a4paper]{article}
\title{AER210---Fluids}
\author{Jonah Chen}

\usepackage[utf8]{inputenc}
\usepackage[margin=0.5in]{geometry}

\usepackage{braket}
\usepackage{physoly}
\usepackage{currfile}
\usepackage{gensymb}
\usepackage{amssymb}
\usepackage{pgf,tikz,pgfplots}
\usepackage{mathrsfs}
\usepackage{textcomp}
\usetikzlibrary{arrows}
\numberwithin{equation}{section}
\pgfplotsset{compat=1.16}

\begin{document}

\sffamily
\maketitle
\tableofcontents

\section{Tutorial 1}

\begin{itemize}
    \item 15.1 \#15
    \begin{align}
        &\int_1^4\int_0^2(6x^2y-2x)\dd y\dd x\\
        &=\int_1^4(12x^2-4x)\dd x\\
        &=(256-32)-(4-2)\\
        &=222
    \end{align}
    \item 15.1 \#29
    \begin{align}
        &\iint_R \frac{xy^2}{x^2+1}\dd A, R=\{(x,y)|0\leq x\leq 1, -3\leq y\leq 3\}\\
        &=\int_0^1\int_{-3}^3\frac{xy^2}{x^2+1}\dd y\dd x\\
        &=18\int_0^1\frac{x}{x^2+1}\dd x\\
        &=18\times \frac{1}{2}\log(x^2+1)\Bigg|_0^1\\
        &=9\log 2
    \end{align}
    \item 15.1 \#31
    \begin{align}
        &\iint_R\ x\sin(x+y)\dd A, R=[0,\frac{\pi}{6}]\times[0,\frac{\pi}{3}]\\
        &=\int_0^{\frac{\pi}{6}}\int_0^{\frac{\pi}{3}}x\sin(x+y)\dd y\dd x\\
        &=\int_0^{\frac{\pi}{6}}x\cos(x)-x\cos(x+\frac{\pi}{3})\\
        &=x\left(\sin x-\sin(x+\frac{\pi}{3})_0^{\frac{\pi}{6}}\right)-\int_0^{\frac{\pi}{6}}\sin x-\sin(x+\frac{\pi}{3})\dd x\\
        &=\frac{\pi}{6}(\frac{1}{2}-1)-\left(-\cos x+\cos(x+\frac{\pi}{3})\right)_0^{\frac{\pi}{6}}\\
        &=\frac{\sqrt 3-1}{2}-\frac{\pi}{12}
    \end{align}

\end{itemize}

\section{Intro to Fluid Flow}
\subsection{Description of Mechanics}
\begin{definition}[Lagrangian Description]
    Describe fluids as a small ``fluid particle'', describe fluids like particles in solid mechanics. i.e. obey euler-lagrange equations.
    \begin{align}
        \mathbf{r}(t)&=(x,y,z)\\
        \mathbf{v}(t)&=(\dot x,\dot y,\dot z)=(u,v,w)
    \end{align}
\end{definition}
\begin{definition}[Eulerian Description]
    Uses density at each points of the fluid in a flow field.
    \begin{align}
        \mathbf v(x,y,z,t)
    \end{align}
    More suitible for analysis of a continum.
\end{definition}
\subsection{Flow visualization}
There are three common concepts used in flow visualization
\begin{itemize}
    \item Streamlines
    \begin{itemize}
        \item A line that is tangent to the local velocity vector at each point at a given instant.
        \item No flow across a streamline.
        \item Can use particle image velocimetry(PIV) to find streamlines experimentally.
        \begin{itemize}
            \item Streamtubes are streamline
        \end{itemize}
    \end{itemize}
    \item Pathlines
    \begin{itemize}
        \item A path a particle takes as it moves.
        \item Experimentally, particle tagged and captured using large exposure.
    \end{itemize}
    \item Streaklines
    \begin{itemize}
        \item A line that connects all the fluid particles that have passed through the same point in space at a previous time
        \item Experimentally, use smoke o
    \end{itemize}
\end{itemize}
\section{Flow Distinction}
\begin{itemize}
    \item A steady flow is when velocity, pressure, temperature, and density is time-independent.
    \item A unsteady flow is when these are time-dependent.
    \item The streamline, streakline, and pathline passing through a particular location will be identical in a steady flow.
    \item Viscous flow regions are regions in which frictional effects are significant.
    \item Invicid flow regions are regions where viscous forces are negligibly small compared to other forces.
    \item 
\end{itemize}


\end{document}