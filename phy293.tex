\documentclass[a4paper]{article}
\title{PHY293---Waves \& Modern Physics}
\author{Jonah Chen}


\usepackage[utf8]{inputenc}
\usepackage[margin=0.5in]{geometry}

\usepackage{braket}
\usepackage{physoly}
\usepackage{currfile}
\usepackage{gensymb}
\usepackage{amssymb}
\usepackage{pgf,tikz,pgfplots}
\usepackage{mathrsfs}
\usepackage{textcomp}
\usepackage{siunitx}
\usetikzlibrary{arrows}
\numberwithin{equation}{section}
\pgfplotsset{compat=1.16}
\everymath{\displaystyle}
\begin{document}

\maketitle
\tableofcontents
\section{Harmonic Oscillators}

\begin{itemize}
    \item Think of simple harmonic motion (SHM) as circular motion projected into one dimension. (a wave is rotation in the complex plane)
    \item One of the simplest system is a mass on the spring, with known force $F=-k\Delta x$. 
    \item For \textbf{SHM}, the motion must be periodic, and the force must be proportional to displacement.
    \item Using Newton's 2nd law on the spring force,
    \begin{align}
        m\ddot x &= -kx\\
        \ddot x + \frac{k}{m} x &= 0\label{12}
    \end{align}
    \item For a mass on the vertical spring, the equilibrium position will be lower due to gravity. 
    \begin{align}
        k(y_1-y_0)-mg&=0\\
        y_1&=y_0+\frac{mg}{k}
    \end{align}
    $y_1$ is the new equilibrium position. SHM will still occur if the system is disturbed. 
    \item Especially when dealing with energy, it is a good idea to have the origin at $y=y_1$.
\end{itemize}

\subsection{The Differential Equation}
\begin{itemize}
    \item In general, the equation for simple harmonic motion is 
    \begin{align}
        \ddot x+\omega^2 x=0
    \end{align}
    \item The solution to \eqref{12} can be represented as \begin{align}
        x=A\cos(\omega t+\varphi_0)
    \end{align} 
    \item $A$ represents the amplitude
    \item $\varphi_0$ is the phase angle (initial phase)
    \item $\omega=\sqrt{\frac{k}{m}}$ is the angular frequency.
    \item The velocity and acceleration can be easily calculated by taking derivatives.
    \begin{align}
        \dot x&=-A\omega\sin(\omega t+\varphi_0)\\
        \ddot x&=-A\omega^2\cos(\omega t+\varphi_0)
    \end{align}
    \item Another way to represent the full solution is
    \begin{align}
        x=a\cos(\omega t)+b\sin(\omega t)
    \end{align}
    where $a=A\cos\varphi_0, b=-A\sin\varphi_0$.
\end{itemize}

\begin{example}
    Determine the amplitude and phase constant of a pendulum moving with a motion described by a sum of two functions, $x_1(t)=0.25\cos\omega t$ and $x_2(t)=-0.5\sin\omega t$.
\end{example}
\begin{sol}
    \begin{align}
        0.25&=A\cos\phi_0\label{1}\\
        -0.50&=-A\sin\phi_0\label{2}
    \end{align}
    Figure out $phi_0$ from the ratio of eq.\eqref{2}/\eqref{1} % complete example
\end{sol}

\subsection{Energy of SHM}
\begin{itemize}
    \item A mass has kinetic energy of $T=\frac{1}{2}mv^2$.
    \item The potential energy is related to the restoring force and by definition,
    \begin{equation}
        \Delta U=-\int F\cdot\dd x=-\int_{x_i}^{x^f}(-kx')\dd x'=\frac{1}{2}k(x_f^2-x_i^2)
    \end{equation}
    \item The conservation of energy here follows from newton's 2nd law.
    \begin{align}
        m\ddot x=-kx % complete derivation at home
    \end{align}
    \item Looking at the potential energy
    \begin{align}
        x&=A\cos(\omega t+\varphi_0)\\
        U&=\frac{1}{2}kx^2=\frac{1}{2}kA^2\cos^2(\omega t)
    \end{align}
    \item Looking at the kinetic energy,
    \begin{align}
        T&=\frac{1}{2}mv^2=\frac{1}{2}m\omega^2A^2\sin^2(\omega t)
    \end{align}
    \item The total energy is
    \begin{align}
        E=T+\frac{1}{2}kA^2=\frac{1}{2}m\omega^2A^2=\frac{1}{2}mv_{MAX}^2
    \end{align}
\end{itemize}
\subsection{Physics of Small Vibrations}
\begin{itemize}
    \item Most system will oscillate with SHM when the amplitude is small. Recall the taylor expansion of an analytic function
    \begin{align}
        f(x)&=f(a)+(x-a)f'(a)+(x-a)^2f''(a)+\dots
    \end{align}
    If $a$ is a minima, $f'(a)=0$. For $x\approx a$, the higher order terms % complete this
    \item The potential energy of a pendulum is \begin{equation}
        U = mgy=mgL(1-\cos\theta)
    \end{equation}
    \item \textbf{For this course, an angle $\theta<\SI{10}{\degree}$, it will be considered small.}
    \item For a simple pendulum,
    \begin{align}
        -mg\sin\theta&=ma\\
        -mg\sin\theta&=m\ddot s, s=L\theta, \dd s=L\dd theta\\
        -g\sin\theta&=L\ddot\theta
    \end{align}
    For small angles, $\theta\approx\sin\theta$
    \begin{align}
        \ddot\theta+\frac{g}{L}\theta=0
    \end{align}
    This is the equation for SHM
    \item The energy of the pendulum is 
    \begin{align}
        E=T+U=\frac{1}{2}mv^2+mg\left(\frac{x^2}{2L}\right)
    \end{align}
    \item For physical pendulum, 
    \begin{align}
        \tau&=I\alpha=I\ddot\theta\\
        -mgd\sin\theta&=I\ddot\theta
    \end{align}
    For small $\theta:\sin\theta\approx\theta$
    \begin{align}
        \ddot\theta+\frac{mgd}{I}\theta=0
    \end{align}
    \item For LC circuits, 
    \begin{align}
        \ddot i+\frac{1}{LC}i=0
    \end{align}
    When a resistor is connected, energy is lost as heat and the circuit behaves like a damped oscilator.
\end{itemize}

\section{Damped Oscillations}

The damped oscillator involves the addition to the drag force that is proportional to $-v$ to the simple harmonic oscillator.

Define
\begin{align}
    \ddot x+\gamma\dot x+\omega^2x&=0\\
    x &= A\exp(\left(\sqrt{\frac{\gamma^2}{4}-\omega_0^2}-\frac{\gamma}{2}\right)t) + B\exp(-\left(\sqrt{\frac{\gamma^2}{4}-\omega_0^2}-\frac{\gamma}{2}\right)t)
\end{align}

For $\omega_0^2\neq\frac{\gamma^2}{4}$.

Critical damping gives the minimum time for the system to return to equilibrium when $\omega_0^2\neq\frac{\gamma^2}{4}$.
\begin{align}
    x=(Ax+B)\exp(-\omega_0 t)
\end{align}

\begin{example}
    A mass $m=3$ is attached to a spring with a value of $k=600$
    \begin{enumerate}
        \item Determine the value of the damping constant $b$ that would produce critical damping.
        \item Determine the value of damping constant $b$ that would decrease the angular frequency by 10\%/
        \item A mass recieve an impulse that ives it a initial velocity $v=2$. What is the maximum resultant displacement and the time when it occurs.
    \end{enumerate}
\end{example}
\begin{sol}
    \begin{enumerate}
        \item \begin{align}
            \omega_0=\frac{\gamma}{2}
        \end{align}

        \begin{align}
            x &= (A+Bt)\exp(-\frac{\gamma t}{2})\\
            \dot x&= \exp(-\frac{\gamma t}{2})\left(B-\frac{\gamma Bt}{2}-A\frac{\gamma}{2}\right)\\
        \end{align}
        $x(0)=0, \dot x(0)=v_i$
        \begin{align}
            x &=v_i t\exp(-\frac{\gamma t}{2})\\
            \dot x &= v_i\exp(-\frac{\gamma t}{2})\left(1-\frac{\gamma t}{2}\right) = 0\\
            t &= \frac{2}{\gamma}=\frac{2m}{b}\\
            x\left(\frac{2}{\gamma}\right)&=v_it\exp(-\frac{\gamma t}{2})=\frac{2v_i}{\gamma e}
        \end{align}
    \end{enumerate}
\end{sol}

\subsection{Energy}
For underdamped oscillator, assume $\omega\approx\omega_0$.
\begin{align}
    E&=T+U=\frac{1}{2}mv^2+\frac{1}{2}kx^2\\
    x&=A_0\exp((i\omega_0-\frac{\gamma}{2})t)\\
    \dot x&=-A_0\exp(t\left(i\omega_0-\frac{\gamma}{2}\right))=A_0\left(i\omega_0-\frac{\gamma}{2}\right)\exp(t\left(i\omega_0-\frac{\gamma}{2}\right))\\
    E&=\frac{1}{2}kA_0^2\exp(-\gamma t)\\
\end{align}

The rate of change of energy is 
\begin{align}
    \dot E=\frac{\dd}{\dd t}\left(\frac{1}{2}mv^2+\frac{1}{2}kx^2\right)=\dot x(m\ddot x+k\dot x)
\end{align}
Recall $m\ddot x=-kx-b\dot x$
\begin{align}
    \dot E=-b\dot x^2=-\gamma E 
\end{align}

\begin{example}
    The energy of a simple harmonic oscillator is ovserved to reduce by a factor of two after 10 complete cycles.
    \begin{enumerate}
        \item How amny cycles will it take to reduce it by a factor of 8?
        \item By what factor would it be reduced after 100 cycles?
    \end{enumerate}
\end{example}
\begin{sol}
    \begin{enumerate}
        \item \begin{align}
            \frac{E}{E_0}&=e^{-\gamma t}\\
            \frac{1}{2}&=e^{-\gamma 10 T}\\
            \left(\frac{1}{2}\right)^3&=\left(e^{-\gamma 10 T}\right)^3=e^{-\gamma 30 T}
        \end{align}
    \end{enumerate}
\end{sol}

\subsection{Quality Factor}

\begin{definition}
    The \textbf{quality factor} is a convenient measurement on how good an oscillator is (how many oscillations it can make before its amplitude would decrease by a certain rate) is defined as 
    \begin{equation}
        Q=\frac{\omega_0}{\gamma}=\frac{\sqrt{km}}{b}
    \end{equation}
    where $\gamma=2\beta=\frac{b}{m}$.
\end{definition}
\begin{itemize}
    \item If $Q=\frac{1}{2}$, the system is critically damped.
    \item Energy $E=E_0\exp(-\gamma t)$.
    \item At two different times $t_1$ and $t_2$ seperated by period $T$, 
    \begin{align}
        \frac{E(t+T)}{E(t)}=\exp(-\gamma T)
    \end{align}
    Which leads to
    \begin{align}
        \frac{E(t)-E(t+T)}{E(t_1)}=1-e^{-\gamma T}\approx -\gamma T\approx\frac{2\pi\gamma}{\omega}=\frac{2\pi}{Q}
    \end{align}
    Thus,
    \begin{align}
        Q=\frac{\text{Energy stored in the oscillator}}{\text{Energy dissipated per radian}}
    \end{align}
    \item We can rewrite the equation of the damped oscillator
    \begin{align}
        \ddot x+\gamma\dot x+\omega_0^2 x&=0\\
        \ddot x+\frac{\omega_0}{Q}\dot x+\omega_0^2 x&=0
    \end{align}
    As $\displaystyle{\gamma=\frac{\omega_0}{Q}, \omega=\omega_0\sqrt{1-\frac{1}{4Q^2}}}$.
    \begin{center}
        \begin{tikzpicture}
            \begin{axis}
                \addplot [
                    domain=0:10,
                    samples=200,
                    color=blue,
                    ]
                    {sqrt(1-0.25*x^(-2))};                
            \end{axis}
        \end{tikzpicture}
    \end{center}
\end{itemize}
\begin{example}
    WHen an electron in $H$ is moved from $n=2$ to $n=3$ states, the atom behaves like a damped oscillator when the lights of frequency \SI{4.57e+14}{\hertz} is emmited. The lifetime of the excited atom is approximately \SI{10}{\nano\second}. What is the value of the quality factor?
\end{example}
\begin{sol}
    As $\gamma=\frac{1}{\tau}, \omega_0=2\pi f$, 
    \begin{equation}
        Q=\frac{\omega_0}{\gamma}=2\pi f\tau=\SI{2.87e+7}{}
    \end{equation}
\end{sol}
\begin{itemize}
    \item RLC circuit:
    \begin{align}
        RI+L\dot I+\frac{q}{C}&=0\\
        L\ddot q+R\dot q+\frac{q}{C}&=0\\
        \ddot q+\frac{R}{L}\dot q+\frac{1}{LC}q&=0
    \end{align}
\end{itemize}
\section{Driven Oscillations}
In the undamped case,
\begin{align}
    m\ddot x+kx&=F_0e^{i\omega t}\\
    m\ddot x+kx&=ka e^{i\omega t}\\
\end{align}
The particular solution is 
\begin{align}
    x=A(\omega)e^{i(\omega t-\delta)}
\end{align} 
where
\begin{itemize}
    \item $\tan\delta=0$ and $A(\omega)=\frac{a}{1-(\omega/\omega_0)^2}$ when $\omega<\omega_0$
    \item $\tan\delta=\pi$ and $A(\omega)=-\frac{a}{1-(\omega/\omega_0)^2}$ when $\omega>\omega_0$
\end{itemize}
\end{document}
