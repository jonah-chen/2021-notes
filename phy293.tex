\documentclass[a4paper]{article}
\title{PHY293---Waves \& Modern Physics}
\author{Jonah Chen}


\usepackage[utf8]{inputenc}
\usepackage[margin=0.5in]{geometry}

\usepackage{braket}
\usepackage{physoly}
\usepackage{currfile}
\usepackage{gensymb}
\usepackage{amssymb}
\usepackage{pgf,tikz,pgfplots}
\usepackage{mathrsfs}
\usepackage{textcomp}
\usepackage{siunitx}
\usetikzlibrary{arrows}
\numberwithin{equation}{section}
\pgfplotsset{compat=1.16}
\begin{document}

\maketitle
\tableofcontents
\section{Harmonic Oscillators}

\begin{itemize}
    \item Think of simple harmonic motion (SHM) as circular motion projected into one dimension. (a wave is rotation in the complex plane)
    \item One of the simplest system is a mass on the spring, with known force $F=-k\Delta x$. 
    \item For \textbf{SHM}, the motion must be periodic, and the force must be proportional to displacement.
    \item Using Newton's 2nd law on the spring force,
    \begin{align}
        m\ddot x &= -kx\\
        \ddot x + \frac{k}{m} x &= 0\label{12}
    \end{align}
    \item For a mass on the vertical spring, the equilibrium position will be lower due to gravity. 
    \begin{align}
        k(y_1-y_0)-mg&=0\\
        y_1&=y_0+\frac{mg}{k}
    \end{align}
    $y_1$ is the new equilibrium position. SHM will still occur if the system is disturbed. 
    \item Especially when dealing with energy, it is a good idea to have the origin at $y=y_1$.
\end{itemize}

\subsection{The Differential Equation}
\begin{itemize}
    \item The solution to \eqref{12} can be represented as \begin{align}
        x=A\cos(\omega t+\varphi_0)
    \end{align} 
    \item $A$ represents the amplitude
    \item $\varphi_0$ is the phase angle (initial phase)
    \item $\omega=\sqrt{\frac{k}{m}}$ is the angular frequency.
    \item The velocity and acceleration can be easily calculated by taking derivatives.
    \begin{align}
        \dot x&=-A\omega\sin(\omega t+\varphi_0)\\
        \ddot x&=-A\omega^2\cos(\omega t+\varphi_0)
    \end{align}
    \item Another way to represent the full solution is
    \begin{align}
        x=a\cos(\omega t)+b\sin(\omega t)
    \end{align}
    where $a=A\cos\varphi_0, b=-A\sin\varphi_0$.
\end{itemize}

\begin{example}
    Determine the amplitude and phase constant of a pendulum moving with a motion described by a sum of two functions, $x_1(t)=0.25\cos\omega t$ and $x_2(t)=-0.5\sin\omega t$.
\end{example}
\begin{sol}
    \begin{align}
        0.25&=A\cos\phi_0\label{1}\\
        -0.50&=-A\sin\phi_0\label{2}
    \end{align}
    Figure out $phi_0$ from the ratio of eq.\eqref{2}/\eqref{1} % complete example
\end{sol}

\subsection{Energy of SHM}
\begin{itemize}
    \item A mass has kinetic energy of $T=\frac{1}{2}mv^2$.
    \item The potential energy is related to the restoring force and by definition,
    \begin{equation}
        \Delta U=-\int F\cdot\dd x=-\int_{x_i}^{x^f}(-kx')\dd x'=\frac{1}{2}k(x_f^2-x_i^2)
    \end{equation}
    \item The conservation of energy here follows from newton's 2nd law.
    \begin{align}
        m\ddot x=-kx % complete derivation at home
    \end{align}
    \item Looking at the potential energy
    \begin{align}
        x&=A\cos(\omega t+\varphi_0)\\
        U&=\frac{1}{2}kx^2=\frac{1}{2}kA^2\cos^2(\omega t)
    \end{align}
    \item Looking at the kinetic energy,
    \begin{align}
        T&=\frac{1}{2}mv^2=\frac{1}{2}m\omega^2A^2\sin^2(\omega t)
    \end{align}
    \item The total energy is
    \begin{align}
        E=T+\frac{1}{2}kA^2=\frac{1}{2}m\omega^2A^2=\frac{1}{2}mv_{MAX}^2
    \end{align}
\end{itemize}
\subsection{Physics of Small Vibrations}
\begin{itemize}
    \item Most system will oscillate with SHM when the amplitude is small. Recall the taylor expansion of an analytic function
    \begin{align}
        f(x)&=f(a)+(x-a)f'(a)+(x-a)^2f''(a)+\dots
    \end{align}
    If $a$ is a minima, $f'(a)=0$. For $x\approx a$, the higher order terms % complete this
    \item The potential energy of a pendulum is \begin{equation}
        U = mgy=mgL(1-\cos\theta)
    \end{equation}
    \item \textbf{For this course, an angle $\theta<\SI{10}{\degree}$, it will be considered small.}
    \item For a simple pendulum,
    \begin{align}
        -mg\sin\theta&=ma\\
        -mg\sin\theta&=m\ddot s, s=L\theta, \dd s=L\dd theta\\
        -g\sin\theta&=L\ddot\theta
    \end{align}
    For small angles, $\theta\approx\sin\theta$
    \begin{align}
        \ddot\theta+\frac{g}{L}\theta=0
    \end{align}
    This is the equation for SHM
    \item The energy of the pendulum is 
    \begin{align}
        E=T+U\frac{1}{2}mv^2+mg\left(\frac{x^2}{2L}\right)
    \end{align}
\end{itemize}
\end{document}