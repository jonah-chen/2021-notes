\documentclass[a4paper]{article}
\title{ESC203---Ethics}
\author{Jonah Chen}

\usepackage[utf8]{inputenc}
\usepackage[margin=0.5in]{geometry}

\usepackage{braket}
\usepackage{physoly}
\usepackage{currfile}
\usepackage{gensymb}
\usepackage{amssymb}
\usepackage{pgf,tikz,pgfplots}
\usepackage{mathrsfs}
\usepackage{textcomp}
\usepackage{setspace}
\usetikzlibrary{arrows}
\numberwithin{equation}{section}
\pgfplotsset{compat=1.16}

\begin{document}
\sffamily
\maketitle
\tableofcontents

\section{Affordances}

Affordance is request, demand, allow, encourage, discourage, refuse. It answers the question of \textbf{how}.

\begin{itemize}
    \item \textbf{Real affordances:} functions attached to a given object---what, potentially, that object affords
    \item \textbf{Percieved affordances:} feature that are clear to the user
\end{itemize}

\subsection{Mechanism and Conditions Framework}
\begin{itemize}
    \item Mechanism: Technology
    \begin{itemize}
        \item \textbf{(request, demand)} initiated by object
        \item \textbf{(encourage, discourage, refuse)} responses to subject inclination
        \item \textbf{(allow)} could be initialted by subject or object
    \end{itemize} 
    \item Conditions: People interacting with technology
    \begin{itemize}
        \item Perceive a range of functions
        \item Having varying skills in operating/interacting (dexterity)
        \item Different level of support due to cultural norms, intelectual regulations.
    \end{itemize}
\end{itemize}

\section{Central Claims of STS Theories}
\begin{itemize}
    \item \textit{Technological Momentum}: Individuals and groups direct the development of new technologies, but investment in large socio-technical systems makes them difficult to change
    \item \textit{Technological Determinism}: the idea that technology develops as the sole result of an internal dynamic, and then, unmediated by any other influence, molds society to fit its patterns
    \item \textit{Social Construction of Technology}: What matters is not technology itself, but the social or economic system in which it is embedded. This maxim, which in a number of variations is the central premise of a theory that can be called the social determination of technology, has an obvious wisdom.
    \item \textit{Actor Network Theory}: 
\end{itemize}
\section{Actor Network Theory (ANT)}
\subsection{Why?}
\begin{itemize}
    \item To analyze sociotechnical systems, in particular organization and power.
    \item More rigorous ways to analyze ever-shifting nature of technology.
    \item ANT attempts to improve \textit{Technological Determinism}, \textit{Technological Momentum} and \textit{Social Construction of Technology} by treating technological and social actors as relational.
    \item Allows us to map affordances.
    \item It is a analytical tool which allows to bring change to a system, not a predictive tool.
\end{itemize}

Political: arrangements of power and authhority in human association as well as a system

\subsection{Types of Actors in ANT}
\begin{itemize}
    \item Human actors:
    \item Conceptual actors:
    \item Artifact actors: Interactions can be mediated
\end{itemize}
The notion of generalized symmetry treats all types of actors as equal in the theory.
\begin{itemize}
    \item Interactions are mediated through non-human actors.
    \item An \textbf{Intermediary} is an actor that transport the force of another actor.
    \item A \textbf{Mediator} is an actor whose outputs cannot be predicted by their inputs.
\end{itemize}
\subsection{Punctualization}
\begin{itemize}
    \item Relationships with affordances is known as \textbf{translation}, Process of making conection and therefore how the technology, system or organization comes to be.
    \begin{itemize}
        \item Actors ``agree'' (resistance must be overcome) that the network is worth building.
        \item Creating convergence between actors
    \end{itemize} 
    \item These questions can only be asked once the network is constructed.
    \begin{itemize}
        \item Processes: How has the translation occured? How is it occuring?
        \item What are the outcomes? How are they ordered?
    \end{itemize}
    \item \textbf{Patterning/Ordering} is a pattern that emerges and is stable enough over time
    \item \textbf{Punctualisation} is when a network of heterogenous bits and pieces with their own roles and resistances is concealed in a coherent entity. All the work of the network is concealed making it hard to detect network conplexities. This is also known as ``Black box''.
    \item Black boxes can be leaky.
\end{itemize}
\subsection{Power}
\begin{itemize}
    \item Neutral: depending on how it is used.
    \begin{itemize}
        \item Originally concentrated in large structures like government or coorperation
        \item Modern sense of power is a component of all relationships between different actors
        \item \textbf{Power is always faced with resistance}
        \item ANT can be used to identify sources of power and suggest ways to dismantle power   
    \end{itemize}
    \item \textbf{You stop depunctualizing when there is an important power relation you want to analyze.}
\end{itemize}

\subsection{Example---Federal Election}

It is helpful to start with one actor and establish relations with other actors. 

affordances
\begin{itemize}
    \item Voter demands pencil
    \item Voter allows voting
    \item Voter requires Voter ID
    \item 
\end{itemize}


\end{document}