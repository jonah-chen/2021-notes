\documentclass[a4paper]{article}
\title{MAT292---ODE}
\author{Jonah Chen}

\usepackage[utf8]{inputenc}
\usepackage[margin=0.5in]{geometry}

\usepackage{braket}
\usepackage{physoly}
\usepackage{currfile}
\usepackage{gensymb}
\usepackage{amssymb}
\usepackage{pgf,tikz,pgfplots}
\usepackage{mathrsfs}
\usepackage{textcomp}
\usetikzlibrary{arrows}
\numberwithin{equation}{section}
\pgfplotsset{compat=1.16}
\begin{document}

\maketitle
\tableofcontents
\section{Newton's Law of Cooling}
Newtons law of cooling states that
\begin{equation}
    \frac{\dd u}{\dd t}=-k(u-T_0)
\end{equation}
Note that there is one trivial solution, the equilibrium solution is $u(t)=T_0$. The meaning of this solution is the temperature of an object doesn't change when it is already at the equilibrium temperature.

\begin{align}
    \frac{\frac{\dd u}{\dd t}}{u-T_0}&=-k\\
    \frac{\dd}{\dd t}\log(u-T_0)&=-k\\
    \log(u-T_0)&=-kt+c_1\\
    u=T_0+\exp(c_1)\exp(-kt)&=T_0+c_2\exp(-kt)
\end{align}
Note that $c_2=\exp(c_1)>0$. However, this is not a complete solution as it cannot describe the solutions with $u<T_0$.

\textbf{Warning:} note that the integral of $\frac{1}{x}$ is $\log|x|$, \textbf{NOT} $\log(x)$. This is what caused the solution to be incomplete.

Hence, $c_2=\pm\exp(c_1)$. Note that $c_1=\pm\infty$ is allowed thus so $c_2$ can take any value.

Below are the integral curves.
% \begin{center}
%     \begin{tikzpicture}
%         \begin{axis}
%             \addplot[domain=0:5]{60-15*exp(x)}
%         \end{axis}
%     \end{tikzpicture}
% \end{center}



\end{document}