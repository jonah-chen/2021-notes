\documentclass[a4paper]{article}
\title{MAT257---Analysis}
\author{Jonah Chen}

\usepackage[utf8]{inputenc}

\usepackage{braket}
\usepackage{physoly}
\usepackage{currfile}
\usepackage{gensymb}
\usepackage{amssymb}
\usepackage{pgf,tikz,pgfplots}
\usepackage{mathrsfs}
\usepackage{textcomp}
\usetikzlibrary{arrows}
\numberwithin{equation}{section}
\pgfplotsset{compat=1.16}

\begin{document}

\maketitle
\tableofcontents
\section{Course Overview}
\begin{itemize}
    \item $\mathbb R\to\mathbb R^n$
    \item Linear Algebra
    \item Continuity
    \item Differentiability
    \item Integration
    \item Key theorem of this class is \textbf{Stokes' Theorem}
    \begin{equation}
        \int_C\dd\omega=\int_{\partial C}\omega
    \end{equation}
    Generalizes the fundamental theorem of calculus:
    \begin{equation}
        \int_{[a,b]}F'(t)\dd t=F(b)-F(a)=\int_{\partial[a,b]}F
    \end{equation}
    Note that $\partial[a,b]=\{b+, a-\}$.
\end{itemize}

\section{Review}
\subsection{Continuity}
\begin{itemize}
    \item Roughly speaking, continuity from $\mathbb R\to\mathbb R$ means if two points are near, their images should be near also.
    \item Thus, in $\mathbb R^n$, the intuitive meaning should be similar.
\end{itemize}


Note there are 2 conventions for $\mathbb R^n$
\begin{enumerate}
    \item The set of all n-dimensional real column vectors.
    \item The set of all n-dimensional real row vectors.
\end{enumerate}
In this class, the distinction is not very important.

\begin{definition}
    For $x, y\in\mathbb R^n$, "The standard (or euclidian) inner product of $x$ and $y$, denoted 
    \begin{equation}
        \langle x, y\rangle = \sum_{i=1}^n x_iy_i
    \end{equation}
    The norm-squared of $x$ is 
    \begin{equation}
        |x|^2=\langle x, x\rangle = \sum x_i^2
    \end{equation}
    and the norm of $x$ is
    \begin{equation}
        |x|=\sqrt{|x|^2} = \sqrt{\sum x_i^2}
    \end{equation}
\end{definition}

\begin{proposition}
    If $x, y, z\in\mathbb R^n$ and $a, b\in\mathbb R$, then
    \begin{enumerate}
        \item The inner product is bilinear \& the norm is ``semi-linear''.
        \begin{align}
                \langle ax+by,z \rangle &= a\langle x, z\rangle + b\langle y, z\rangle\\
            \langle z, ax+by \rangle &= \dots\\
            |ax|&=|a||x|
        \end{align}
        \textbf{Aside}: 
        \begin{equation}
            1=\sqrt 1=\sqrt{-1\cdot-1}=\sqrt{-1}\sqrt{-1}=i\cdot i=-1
        \end{equation}
        \item \begin{equation}
            |x|\geq 0 \& |x|=0\iff x=0
        \end{equation}
        \item \begin{equation}
            \langle x,y\rangle = \langle y,x \rangle
        \end{equation}
        \item \textit{Cauchy-Schwarz inequality}
        \begin{equation}
            |\langle x,y\rangle|\leq|x||y|
        \end{equation}
        with equality if $x\&y$ are dependent. 
        \item \textit{Triangle inequality}
        \begin{equation}
            |x+y|\leq |x|+|y|
        \end{equation}
        \item \textit{Polarization identity}
        \begin{equation}
            \langle x,y\rangle = \frac{|x+y|^2-|x-y|^2}{4}
        \end{equation}
    \end{enumerate}
\end{proposition}
\begin{proof}
    \begin{enumerate}
        \item $|x|=\sqrt{\sum x_i^2}$
        $|x|=0\implies\sum x_i^2=0\implies\forall i, x_i^2=0\implies \forall i,x_i=0\implies x=0$
        \item 
        For $s,t\in\mathbb R^n$
        \begin{equation}
            |s+t|^2=|s|^2+|t|^2+2\langle s,t \rangle
        \end{equation}

        Look at 
        \begin{align}
            0\leq\Big||y|^2x-\langle x,y\rangle y\Big|^2&=|y|^4|x|+\langle x,y\rangle^2|y|^2-2|y|^2\langle x,y\rangle^2\\
            &=|y|^2\left(|y|^2|x|^2-\langle x,y \rangle^2\right)
        \end{align}
        This is equal to zero only if $|y|^2x-\langle x,y\rangle y=0$. If we have equality, that implies $x\&y$ are dependent.
        \textbf{Why, what does this mean?}

    \end{enumerate}
\end{proof}



\end{document}
